\documentclass[12pt,a5paper]{book}
\usepackage[utf8]{inputenc}
\usepackage[spanish]{babel}
\usepackage{titlesec}
\usepackage[left=2cm, right=2cm, top=3.5cm, bottom=3cm]{geometry}
\usepackage{xcolor}
%\usepackage{hyperref}
\usepackage{multicol}


%\usepackage{incgraph}
%\usepackage[all]{genealogytree}

\clubpenalty=10000
\widowpenalty=10000

\usepackage{verse}
\usepackage{lipsum}

\newcommand*\justify{%
  \fontdimen2\font=0.4em% interword space
  \fontdimen3\font=0.2em% interword stretch
  \fontdimen4\font=0.1em% interword shrink
  \fontdimen7\font=0.1em% extra space
  \hyphenchar\font=`\-% allowing hyphenation
}

\titleformat{\section}{\center\LARGE\itshape}{\thesection}{1em}{}
\titleformat{\subsection}{\Large\itshape}{\thesection}{1em}{}
\titleformat{\subsubsection}{\normalfont\itshape}{\thesection}{1em}{}

\title{\textbf{ \Huge La Historia de mi Vida}}
\author{\textsc{Diario de Marina Gardía González}\\ {\footnotesize \textit{San Bartolomé}, \textsc{Tui}}}
\date{\textit{26 de junio, 1946}}

\begin{document}

\maketitle


\[\]
\pagebreak



\begin{center}
\textsc{As de cantar que ch'ei de dar zonchos;}\\
\textsc{as de cantar que ch'ei de dar moitos.}
\end{center}

\begin{multicols}{2}
\textit{``As de cantar,}\\
\textit{Meniña gaiteira,}\\
\textit{As de cantar,}\\
\textit{que me morro de pena.}\\

\textit{Canta, meniña,}\\
\textit{Na veira da fonte,}\\
\textit{Canta, dareiche}\\
\textit{Boliños do pote.}\\

\textit{Có son da gaitiña}\\
\textit{Co son da pandeira,}\\
\textit{Che pido que cantes,}\\
\textit{Rapaza morena.}\\

\textit{Có son da gaitiña,}\\
\textit{Có son do tambor,}\\
\textit{Che pido que cantes,}\\
\textit{Meniña, por Dios.''}
\end{multicols}
\begin{flushright}
--- Rosaía de Castro
\end{flushright}

% Continuar aquí: \href{https://docs.google.com/document/d/1DGDB_lBm7ChNLynKc1PRi79eT4zxSLuu0hF1sMW_CrY/edit}


\pagebreak


\section*{I.\\1945 - 46, orígenes}

Mi nombre es Marina García González. Nací el 26 de junio de 1946 en la Calle Calzada de San Bartolomé, Tui; antes era el número 13, pero después le pusieron el N$^\circ$ 28. Mis padres fueron Pedro García González y Marina González Martínez. 

Mi padre era manchego nacido en Puertollano, provincia de la Ciudad Real, al que él llamaba ``\textit{El pueblo de las dos mentiras, porque no tiene puerto ni es llano}''. Vino a hacer la mili a Tuy, y para llegar tuvo que hacerlo caminando desde la estación de Guillarei hasta el cuartel que estaba en la calle Santo Domingo. Delante de mi casa había un pequeño poste y él puso allí la maleta para descansar un rato, y dijo: “Aquí me voy a  quedar”, y aquí se quedó. Se enamoró de mi madre y se casaron el 24 de septiembre de 1945. Cuando yo nací mi madre falleció, el 26 de Junio de 1946. Mi padre se volvió a casar con otra mujer el 24 de diciembre de 1948.

Mi mamá murió de parto. Tenía la albúmina alta, ya que cuando se encontraba mal le daban sopas de burro cansado, como se les solía llamar, que son sopas de vino caliente. Cuando el hígado no funciona bien, la concentración de albúmina en sangre puede ser peligrosa, y posiblemente esto haya terminado con su vida y casi con la mía.

Al nacer en esas condiciones, yo era tan pequeñita y delicada que cuando me vio Matilde, la madrina de mi hijo Jose Antonio, le dijo al médico don José:

---No sé si le haríamos un favor metiéndola en la caja con su madre.

A lo que el médico le contestó:

---Ay mujer, no diga eso. Se nos fue la madre así que hay que salvar a la hija; lo que siento yo es que esta pobre mujer no va a ser capaz porque ya es mayor--- refiriéndose a la madrastra de mi madre, María Bugallo López.

---Por eso no se preocupe doctor, que de la niña me encargo yo--- contestó Matilde armándose de valor.

---Pues si usted se encarga de la niña, con nuestra ayuda y la de Dios la sacaremos adelante--- dice Don José Jurado Romero.

La comadrona que nos atendió en el parto junto con don José Jurado Romero se llamaba doña María Dorado.

Yo me quedé a cargo de María, la madrastra de mi madre, pero como ella era mayor y nunca tuvo hijos no sabía cómo cuidarme. Matilde, que era vecina de mi madre, sabía cuidar niños ya que a pesar de ser soltera, había tenido muchos hermanos. Entonces Matilde ayudó a María y entre las dos me sacaron adelante.

María no quería a mi padre en casa porque tenía problemas con el alcohol. Así que lo mandó fuera y él se fue a vivir para las calles de abajo, concretamente la calle San Telmo. Allí se enamoró de una mujer del Seixal ---que también se llamaba María---, se casaron y se fueron para Castilla.

\section*{II.\\1953, niñez}

A los 7 años vivía en San Bartolomé. En esa época hice la comunión y asistía a las clases de doña Eufemia Halcón. Estas clases se impartían a las niñas del barrio en una casa al lado de la iglesia de San Bartolomé.

Al poco tiempo murió María Bugallo López, la madrastra de mi madre que ejerció como cuidadora mía hasta el final de sus días. Para no quedarme sola me fui a vivir con mi tío Pepe\footnote{Mi tío Pepe es el padre de las argentinas Laura y Ana María.}, hermano de mi madre. Pepe estaba casado con Ana, que era portuguesa, y tenían dos hijas llamadas Laura y Ana María, primas mías que tiempo después emigraron a Argentina. 

Mi padre tuvo un accidente de trabajo y se rompió una pierna. Aprovechando que estaba de baja vino de vuelta a Galicia. Cuando llegó se enteró de que había muerto \textit{la vieja}\footnote{\textit{La vieja} es María Bugallo López, la madrastra de mi madre.} y yo estaba mal. Entonces le escribió a su mujer, le pidió que le diera de baja en la empresa, que arreglase otras cosas y que se viniera también para Galicia. Con ella se vino el tío Miguel, hermano de mi padre que estaba soltero y vivía con ellos.

Se instalaron en mi casa\footnote{La casa de tres pisos en la que ahora vive la tía Marina.}. Esa casa la heredé de mi abuelo Manuel González Rodríguez, padre de mi madre. Mi abuela por parte de madre se llamaba Filomena Martínez Gózmez. Cuando mi abuelo repartió el capital yo fui la heredera, ya que mi madre había fallecido. Por parte de mi padre, mi abuelo era Elías García García y mi abuela María del Carmen González Sánchez. \textcolor{purple}{[pedir a Lucas árbol genealógico]}.

Fuimos a vivir a O Seixal porque la madre de mi madrastra estaba enferma (María Estévez Vilar mi madrastra, Emilia Vilar González la madre de mi madrastra). Mi madrastra, que también se llamaba María y era la mujer de mi padre, tenía que cuidar a su propia madre (la señora Emilia) en O Seixal. Estaba enferma de cáncer y por eso fuimos para allí.

Cuando estaba en O Seixal iba al colegio de las Doroteas, en Río Muíños. Ahí estuvimos casi dos años. Murió doña Emilia cuando yo tenía 9 años y entonces fue cuando volví a San Bartolomé. Al tratar de volver al colegio de doña Eufemia, ella no me quería alegando sin mucha coherencia: ``Ya estuviste aquí hace tiempo y ahora queres volver''. Finalmente me aceptó.

Una vez cumpliese los 12 años doña Emilia me mandaría para casa de nuevo diciendo que ya sería muy mayor para el colegio. En mi misma clase siempre hubo algunas niñas mayores que yo, pero yo sería la única a la que mandaría para casa a trabajar. En esa época entrabas en el colegio a los 6 años y ya te encontrabas en tu misma clase con niñas de 8 y 10 años.

%------

%A los 7 años viví en San Bartolomé, y ahí estuve muy poco tiempo. Luego fuimos a O Seixal y con 9 años, al volver a San Bartolomé, la profesora no me quería alegando sin mucha coherencia: ``Ya estuviste aquí hace tiempo y ahora queres volver'', \textcolor{blue}{pero finalmente me aceptó (?)}.
%
%Con 7 años hice la comunión y empecé a ir al colegio en San Bartolomé. Durante esa breve temporada asistí a las clases de doña Eufemia Halcón, en una casa al lado de la iglesia. Cuano nos fuimos a vivir a O Seixal fue para cuidar a doña Emilia, la madre enferma de mi madrastra María Estévez.
%
%Cuando estaba en O Seixal iba al colegio de las Doroteas, en Río Muíños. Ahí estuvimos casi dos años. Murió doña Emilia cuando yo tenía 9 años y entonces fue cuando volví a San Bartolomé con mi tío Pepe\footnote{Mi tío Pepe es el padre de las argentinas Laura y Ana María.}, hermano de mi madre. Pepe estaba casado con Ana, que era portuguesa, y tenían dos hijas llamadas Laura y Ana María, primas mías que tiempo después emigraron a Argentina. \textcolor{blue}{En ese momento iba al colegio de doña Eufemia (la segunda vez que estuviste en San Bartolomé también fuiste al colegio de doña Eufemia?)}. 
%
%Mi padre tuvo un accidente de trabajo y se rompió una pierna. Aprovechando que estaba de baja vino de vuelta a Galicia. Cuando llegó se enteró de que había muerto \textit{la vieja}\footnote{\textit{La vieja} es María Bugallo López, la madrastra de mi madre.} y yo estaba mal. Entonces le escribió a su mujer, le pidió que le diera de baja en la empresa, que arreglase otras cosas y que se viniera también para Galicia. Con ella se vino el tío Miguel, hermano de mi padre que estaba soltero y vivía con ellos.
%
%Se instalaron en mi casa\footnote{La casa de tres pisos en la que ahora vive la tía Marina.}. Esa casa la heredé de mi abuelo Manuel González Rodríguez, padre de mi madre. Mi abuela por parte de madre se llamaba Filomena Martínez Gózmez. Cuando mi abuelo repartió el capital yo fui la heredera, ya que mi madre había fallecido. Por parte de mi padre, mi abuelo era Elías García García y mi abuela María del Carmen González Sánchez. \textcolor{purple}{[pedir a Lucas árbol genealógico]}.
%
%Al poco tiempo fuimos a O Seixal porque la madre de mi madrastra estaba enferma (María Estévez Vilar mi madrastra, Emilia Vilar González la madre de mi madrastra).
%
%Se instalaron en mi casa que había heredado de mi abuelo y poco después fuimos a O Seixal. Heredé esta casa de mi abuelo por parte de mi madre, llamado Manuel González Rodríguez. Mi abuela era Filomena Martínez Gózmez. 
%
%Mi madrastra, que también se llamaba María y era la mujer de mi padre, tenía que cuidar a su propia madre (la señora Emilia) en O Seixal. Estaba enferma de cáncer y por eso fuimos para allí. [¿Dónde vivías antes de que viniese tu padre de Castilla con tu madrastra? En la casa en la que ahora vive la tía Marina].


\subsection*{Comunión}

A los 7 años hice la Primera Comunión, un jueves de junio de 1953 en \textit{Corpus Christi}. Por aquel entonces el Corpus se celebraba los jueves, pero a partir de 1989 pasó a celebrarse los domingos. Como decía la madrina de José Antonio:
\begin{center}
\textit{``Hay tres jueves en el año que días santos son: jueves Santo, Corpus Christi y jueves de la Ascensión''.}
\begin{flushright}
  — Matilde.
\end{flushright}
\end{center}

Mi Primera Comunión la celebramos en la iglesia de San Bartolomé como no podría ser de otra manera, ya que era allí donde vivíamos. Fui la única niña que comulgó ese día. Después de la misa se hizo una comida normal en mi casa. En ese momento vivíamos en la casa de tres pisos en la que ahora, 2023, vive mi hija Marina.

No me hicieron ningún regalo en la comunión, ya que de aquella eso no era común entre las familias como nosotros. Fui con un vestido prestado de Marita de Ferruxo. Su madre, la señora Teresa, me dijo que el de Marisol no me lo prestaba porque era muy bueno, entonces me dejó el de Marita que era más ordinario.


\subsection*{Colegios}

En aquella época nos separaban niños y niñas en aulas distintas, ya que también recibíamos distintas educaciones. En muchas ocasiones incluso las aulas estaban en edificios separados. 

Teníamos clase por las mañanas y por las tardes. Por las mañanas posiblemente tanto niñas como niños aprenderíamos las mismas asignaturas. Aprendíamos lengua haciendo dictados, donde la profesora iba escribiendo en la pizarra mientras nosotras copiábamos; matemáticas haciendo cuentas; geografía recitando los nombres de los ríos...

Por las tardes en cada aula se aprendían tareas distintas. Yo no sé qué aprenderían los niños, pero a nosotras nos enseñaban labores de casa como coser, hacer bordados o cuidados domésticos en general. Yo hacía vainica, que consiste en deshilachar la tela para a continuación ir cosiendo los hilos restantes para hacer patrones.

%En aquella época las aulas de los colegios estaban separadas niñas y niños, ya que recibíamos educaciones distintas [en todos los colegios? Sí, pero en Doroteas solo los ricos][qué aprendía cada uno]:





\subsubsection*{Colegio de las Doroteas}

En O Seixal estuvimos casi dos años hasta que la Señora Emilia falleció. Durante ese tiempo iba al colegio de las Doroteas.

Este colegio tenía dos aulas. Los niños de familias pudientes iban al aula situada en el convento de las Doroteas en Tuy, ya que era una enseñanza de pago. Mientras que los que no teníamos dinero íbamos a Río Muíños. Nosotros les llamábamos el colegio de los ricos y el colegio de los pobres. Ambos eran de las Doroteas, pero estaban separados.

Estoy segura de que también ricos y pobres recibiríamos una educación muy distinta. Al menos los ricos tendrían unas comodidades muy superiores a las nuestras y una dedicación mucho más esmerada por parte de los profesores.

Para ir al colegio, yo bajaba por A Videira, un camino empedrado que iba a dar a La Tafona, que era otro camino que finalmente iba a dar a Río Muíños. Así era todos los días.


\subsubsection*{Colegio de Doña Eufemia Halcón}

Tras el fallecimiento de doña Emilia nos vinimos a vivir a San Bartolomé, así que volví al colegio de doña Eufemia, del cual ya os hablé.

En este colegio tenía muchas amigas, pero también amigos. Hoy cuando nos vemos por la calle todavía nos saludamos. Aunque estábamos en clases separadas, cuando salíamos al recreo jugábamos todos juntos. Al terminar el tiempo de jugar, los profesores decían ``¡Arriba!'' e íbamos cada uno para su clase. 

Teníamos clase por la mañana y por la tarde todos los días menos los jueves por la tarde. Los que íbamos al colegio de San Bartolomé éramos todos vecinos del barrio. Las hijas e hijos de la familia Gregores-Rocha de San Bartolomé venían al colegio de Doña Eufemia. Concha, sus hermanas Pitusa, Trini, \textcolor{orange}{Pilucha} y sus hermanos Paco, Moncho, Pepe, Mingos \textcolor{orange}{y Esteban}. Ellos estaban en la clase de los niños \textcolor{orange}{(Esteban y Pilucha también eran sus hermanos)}.

Una hija de Concha, Mari Cruz, se casó con un hijo mío, Javi. Mira tú lo que son las cosas. Sin querer fuimos a emparentar.

Este colegio como ya dije estaba separado en la parte masculina y la femenina. La profesora de las niñas era doña Eufemia y el de los niños era don Martín Borrego, su marido. El aula de las niñas está situada donde hoy en día está la casa de los zamoranos, al lado de la iglesia de San Bartolomé. En la planta baja estaba la tienda de Valeriano, un ultramarinos. En la planta del medio estaba el aula al que íbamos a clase las niñas. Arriba de todo vivían los dueños de la casa: La señora Olivia y el Señor Celestino, que tenían dos hijos: uno se llamaba Celestino y el otro Manolo.

El colegio de los niños, donde impartía las clases don Martín Borrego, se encontraba en un edificio haciendo esquina justo donde comienza la caye Arrayal, que es una de las casas que tiraron y ahora es una finca.


\subsubsection*{Historia de Elvira, la niña de las zapatillas rotas}

En este colegio tenía muchos amigos, chicos y chicas. Había un grupo de ellas que se tenían por más superiores a otras. Había una chica de familia humilde, más de lo común, que se llamaba Elvira. En su familia eran 9 hermanos y vivían como podían. Elvira  quería ir de paseo con las otras niñas, pero como no podía permitirse renovar las zapatillas rotas no la quisieron aceptar en el grupo. También se acercó a mí para ver si querría salir con ella, y como a mí no me importaba que tuviese ropa en mal estado, le dije que sí. Por supuesto que salimos a pasear juntas.

Cuando fue mayor y faltaron sus padres, ella se fue para Barcelona a trabajar. Estuvo allá unos años y después volvió a Galicia. Ahora pasa delante mía ya no me conoce. Si yo no le hablo, ella no me habla. Ya no se acuerda de lo de antes, pero yo no me olvido ni de sus hermanos ni de sus padres, como tampoco me olvido de que nosotros éramos pobres también y yo misma andaba con los calcetines remendados con plantillas y no me da vergüenza decirlo.

En aquella época también se remendaba la ropa de vestir, la ropa de la cama, y lo que hiciese falta. Calzábamos chamangas, andábamos con los pies calientes y secos, ¡y tan contentos!


\subsubsection*{Historia del campo de castaños}

En el colegio de doña Eufemia había dos hermanas mellizas, Carmiña y Milagros, que eran vecinas mías. El campo de su abuela estaba detrás de la iglesia de San Bartolomé y en él tenían algnos castaños. 

Un día estábamos en el recreo del colegio y me dijeron ``Marinita, ¿vamos a las castañas de mi abuela?'', le dije que sí y entonces fuimos por detrás de la iglesia. Una de ellas me dice que baje yo primera por el terraplén, así lo hice y nada más llegar abajo estaba su abuela, la Señora Ermelinda, detrás de un castaño con una vara del propio arbol en la mano. Me dio unos latigazos por las piernas que me dolieron un montón. Fueron a llamar a mi tía y me vino a buscar. Entonces pensé que mi tía me pegaría más, pero no fue así. Me trajo para casa y vio como tenía las piernas llenas de cardenales. Me puso paños de vinagre y me curó. No volvieron a decirme más veces para ir a las castañas, pero seguimos siendo amigas. 

Milagros se casó con Juan, un chico de Tui y tuvieron tres hijos. Carmiña se casó con Daniel que también era de Tui y tuviern un hijo. Su otra hermana, Pilar, tambén se casó y no quiso tener hijos. Ahora Milagros murió, Carmiña y su familia se fueron a vivir a Vigo y Pilar sigue viviendo aquí, donde vivieron siempre, ya que sus padres murieron y ella fue la que se encargó de cuidarlos.



\section*{III.\\1948, preadolescencia}

Cuando no iba al colegio por la mañana, hacía las cosas de casa. Había ocasiones en las que en casa me decían que no podía ir al colegio porque tenía que atender otras labores. En esas ocasiones iba a la plaza a comprar pescado, luego hacía la comida porque mi tía iba a trabajar al campo para alguna vecina. También lavaba ropa de las vecinas porque mi padre trabajaba en la fábrica de Serrerías del Miño\footnote{Esta fábrica es donde después fue Canfrán.}. 

Estuve en el colegio de San Bartolomé hasta los 12 años, que me fui porque Doña Eufemia dijo que ya era mayor para estar en el colegio y me mandó para casa.

Desde que vino para casa María Estévez, mi madrastra, yo nunca fui feliz pues me trataba todos los días a golpes. No sabía mandar hacer las cosas si no era golpeando. He de decir que alimento no me faltaba, pero palo tampoco.

Con 14 ó 15 años me mandaba al molino a llevar maíz, a veces para moler, otras para vender. En esa época podía llegar a cargar sacos de entre 15 y 20 kg.

Mi padre también trabajó en la construcción. Era muy trabajador, pero tenía una enfermedad y es que le gustaba mucho el vino. Cuando cobraba ya no venía por casa, sino que se metía en la taberna y gastaba parte del jornal. Luego le daba a mi tía 100 ptas. y él se quedaba con el resto.

Con 100 ptas. no llegaban para comer 3 personas toda la semana, así que mi tía tenía que trabajar fuera. Cuando todavía iba al colegio por la mañana, a la tarde iba a junto de mi tía, le ayudaba en el campo y luego los dueños del campo en el que se estuviese trabajando en ese momento nos daban la merienda: un cacho de pan con lo que fuera, pescado frito, tortilla, o cualquier sobra del medio día que le quedara.

Mi padre fue a trabajar a un almacén de patatas, El Prado. Camilo Padro Arias se llamaba el dueño. Yo iba a ayudarle a llenar los sacos y a coserlos. A mí no me pagaban nada. Camilo estaba casado con Rita y tenían un niño pequeño, entonces como necesitaban que alguien cuidase al niño, me dijeron a mí de hacerle recados, plancharle la ropa o lo que hiciera falta. Ahí sí que me pagaban y me daban 100 ptas. al mes.


\subsubsection*{Me escapo de casa porque mi madrastra me pega}

Desde que mi madrastra vino a vivir con nosotros, nunca fui feliz, como ya mencioné a lo largo de esta historia. Ella me trataba muy mal, me pegaba y no me quería.

Un día yo estaba haciendo las cosas de casa y dije para mis adentros: ``Como me pegue hoy, me marcho de casa'', y así fue. No sé por qué la emprendió y me pegó. Después me mandó a un recado y yo ya no volví para casa. Me fui hacia el Puente Internacional. Al llegar allí y ver aquel puente, pensé ``¿y esto a dónde me llevará?'', entonces di la vuelta y fui por otra carretera dirigiéndome a Las Bornetas. Allí paré en una tienda de ultramarinos. Todos los vecinos se enteraron de lo que pasaba al verme allí, ya que les conté que me había ido de casa.

Alguien de mi familia daría el parte de que yo había desaparecido, entonces se plantó allí el guardia municipal, llamado Rucho. Me trajo para la comisaría y me tomaron declaración de lo ocurrido. A continuación me mandaron para casa. Después mi tía me decía: ``ya tienes una ficha en la policía, como una delincuente''.



\section*{IV.\\1950, primeros trabajos fuera de casa}

A los 14 años fui a cuidar a un niño llamado Jose Manuel y ganaba 100 pesetas al mes. Ese niño es hoy el que tiene el almacén de patatas Prado. Los padres se llamaban Camilo y Rita. 

Después fui a cuidar Jose Ángel Fon y también me pagaban 100 pesetas al mes. Al pagarme el jornal a final de mes y llegar a casa, mi tía ya estaba esperando con la mano abierta para que le diese el dinero, a mí no me permitía quedarme ni con una peseta. Cuando Jose Ángel aprendió a caminar ya no me necesitaban, así que volví para casa. Hoy en día trabaja en Radio Tui.

Con 16 años mpecé a repartir pan en el horno de las Olivas, en la del ``medio metro''. Lo llamaban así porque Manuel el dueño era muy bajito. En este horno estuve un año trabajando, que estaba en Calle Calzada y hoy en día ya tiraron abajo el edificio.

Después de ahí fui a la panadería del Rollo donde estuve otro año. Salí de ahí para casarme.


\section*{V.\\1963 - 64, noviazgo y boda}


El 4 de agosto del 1963 fui a la fiesta de la Virgen del Camino con mi madrastra, que yo le llamaba tía. Era la subida de la bandera. Allí estábamos viendo como tocaba la banda de música y vi venir a un chico hacia nosotros. Pensé para mis adentros: ``Ahí viene el cuco'', que era Antonio.

Cuando eran las fiestas, siempre íbamos juntas mi tía y yo, ya que mi padre iba a su aire. Él se divertía a su manera, tomando unos vasos de vino por ahí y luego venía contento para casa cantando algunas canciones muy bonitas.

Yo conocía a Antonio de verlo pasar por aquí con el carro de bueyes hacia el campo del Cabildo, a él y a su hermana. Ella iba delante de los bueyes, y él montado encima del carro... mira qué listo era.

Esa noche, que fuimos a la fiesta, me sacó a bailar y ya quedó allí con nosotros. Luego nos vino a acompañar a casa. A partir de ese día, empezó a venir los domingos a buscarme a casa para salir de paseo. Después venía también los sábados, los jueves... hasta que empezó a venir todos los días y ya hasta entraba en casa.

Cuando salíamos de noche al cine o a la fiesta mi tía tenía que venir con nosotros, no me dejaba salir sola con él de noche. Cuando venía Antonio a buscarme, la primera persona que salía a la puerta era ella. Entonces una vecina, la Señora María ``Grila'' decía: ``Antonio. Quen é a túa moza, a Marina ou a María?''.

Estuvimos de novios siete meses. A los siete meses nos casamos, un 8 de Marzo del 1964. La boda fue a las 12 del mediodía en la capilla de los Maristas, porque Antonio trabajaba allí. Hicimos la comida en casa. Esto fue un domingo, y por la tarde nos fuimos a Vigo con el hermano Enrique era el director de los Maristas, nos llevó a ver el monte del Castro y luego nos quedamos en casa de una amiga suya que se llamaba Doña Benedicta. Allí estuvimos hasta el martes que volvimos para casa. 

Nos casamos en los Maristas por que Antonio trabajaba allí y el Director, que era el hermano Enrique, quiso que así fuera. El hermano Enrique fue el que organizó todo. Quería casarnos en la capilla de Os Maristas, pero este es un colegio de noviciados entonces en esa capilla en principio hacían misas para ellos mismos y no para el resto del público. Enrique fue a hablar con el cura de San Bartolomé, pero le dijo que eso no era de su jurisdicción, sino que esa decisión le correspondía tomarla al Obispo de la diócesis Tui-Vigo. Finalmente, hablando con el Obispo en Vigo, este le dio permiso para celebrar la boda en la capilla de Os Maristas y este arregló todo lo que fue necesario.


\section*{Alcumes}


Mi abuelo era Manuel \textit{do Poste}. Al parecer tenía la costumbre de mover los marcos, esa práctica tan conocida en la Galicia rural. A los marcos por aquí también se le llamaban postes, y así le quedó el mote. ``Tenían una pelea con el poste, poniéndolo pa'quí y pa'llá, pero al final quedó donde tenía que quedar''.

Mi suegro era José \textit{o Cuco}. Supongo que era demasiado listo o astuto. Recuerdo haberle preguntado varias veces, pero él contestaba ``Eran cousas dos rapaces, non sei non me acordo'', osea que no se quería acordar.


\section*{VI.\\1965 - 69, nacimiento de los hijos varones.}

% Continuar aquí

El Me quedé embarazada enseguida, pero a los 3 meses tuve un aborto. Me quedé emabarazada otra vez y el 8 de abril del 1965 nació mi primer hijo José Antonio.

Por ser el primero el parto fue un poco duro, pero salió bien gracias a Dios. Cuando nos casamos, los padrinos de la boda iban a ser Saladina, la hermana de Antonio, y su marido José María Rocha. Les pertenecía ser los padrinos del niño, pero como mi cuñado dijo que el niño tenía que llevar su nombre y nosotros le pusimos Jose Antonio, no quisieron ser los padrinos. Entonces, tuvimos que buscar otros padrinos, y fue mi padre Pedro García Gonzalez y Matilde Moldes Martinez\footnote{La que ayudó a criarme, pero que al principio quería meterme en la caja con mi madre}. 

Enseguida me quedé embarazada otra vez y el 27 de Febrero de 1966 nació Juan Manuel. Antonio quiso ponerle Juan Manuel porque que eran los nombres de su hermano y de mi abuelo, del que heredé la casa. Entonces mi padre Pedro se enfadó porque quería que le pusieran su nombre; él decía ``A mi ya no me quieren, me desprecian, no quieren mi nombre'', pero a pesar de eso, así quedó la cosa. Los padrinos de Juan Manuel, que hoy en día le llamamos Noli, fueron Horacio y Leonor, que son el hermano de mi madrastra y su señora.

Me quedé embarazada otra vez y el 22 de Abril de 1967 nació Pedro Roger. En dos años me junté con tres niños. Como le pusimos Pedro, el nombre de mi padre y su padrino (Juan Roger Fon), mi padre ya se puso contento porque llevaba su nombre. Los padrinos fueron Juan Roger (que le llaman “Roxé”) y su hermana María Monsterrat. Cuando lo bautizamos, Roger nos dijo ``No le llameis Roger, llamadle Pedro, que a mi me empezaron llamando Roger, y así me quedó''. Pues a él no lo llamamos Roger, le llamamos Xe que es más corto. Pues ahora, en dos años tuve tres niños.

 El 27 de Agosto de 1969 nació Francisco Javier. Cuando Antonio lo iba a registrar, Matilde le dijo ``Qué nombre le vas a poner al niño?'' y le dijo ``Le voy a poner Javier'', a lo que respondió ``Por qué no le pones Francisco Javier?'', y le dijo ``Non quero Franciscos, Javiero sólo!''. Así quedó la cosa. Cuando fuimos a bautizarlo a la iglesia, el cura le dijo ``Tiene que ser Francisco Javier, porque a mi me llaman Javier pero soy Francisco Javier y para eso hay San Francisco Javier, que es el 3 de Diciembre''. Entonces así quedó. Los padrinos de Francisco Javier fueron Moncho y Flora, los sobrinos de mi madrastra.


\section*{VII.\\1970, Antonio se va a Alemania.}

Pasé mi juventud criando niños. El 10 de marzo de 1970 se marchó Antonio para Alemania y yo me quedé aquí con mi tía, mi padre y los niños atendiendo la casa y las tareas del campo. En ese momento yo ya tenía cuatro hijos.

Teníamos animales y trabajábamos el campo. Por la mañana levantaba a los mayores para ir al colegio. Al campo íbamos mi padre, mi madrastra y yo. Los niños se los dejaba a Matilde y ella me los cuidaba. Los mayores empezaron a ir al colegio a los 6 años.

Siempre les decía que cuando llegase Papá de alemania les iba a traer un regalito, pero cuando venía, Javi lloraba porque no lo reconocía. Se ponía contento al decirle que iba a venir Papá, pero al verle se asustaba porque no estaba acostumbrado. Le costó acostumbrarse a él. Los otros se pusieron todos contentos porque les trajo un tren eléctrico y otros regalos. \textcolor{blue}{[Preguntar a los tíos los regalos que les trajo el abuelo de Alemania]} 

Venía de Alemania más o menos una vez cada seis meses, en verano y en Navidad. Se quedaba en casa unos tres meses antes de volver a atravesar Europa.

Cuando Antonio vino, lo hizo porque tenía miedo de hacer como otros hombres, que se liaban con alguna mujer y ya no volvían con la familia a Galicia. Al instalarse de nuevo en Galicia encontró trabajo y se puso manos a la obra. En total estuvo en Alemania cinco años, y muchos de sus compañeros no volvieron porque se olvidaron de la mujer, de los hijos y de todo.

Trabajó en la construcción, en la de Oliveira. Era una granja, así que se dedicaba a atender a los animales y a labrar el campo.


\subsection*{La primera televisión}

Uno de los viajes que Antonio vino a Galicia desde Alemania, compró aquí una televisión. En aquel momento algunas familias empezaban a tener televisión en casa, pero muy pocas.

Los niños iban a ver la TV a la tienda de Antonio, pero allí iban hombres que hablaban de cosas obscenas y decían palabrotas. De pequeños son como esponjas y copian todo lo que ven y oyen, por eso decidió que había que comprar una TV para casa. Todavía vivíamos en la casa de tres pisos que heredé de mi abuelo, así que la instaló en el descansillo de las escaleras. Todos los niños del barrio se sentaban en los escalones para ver la TV como si fuesen las bancadas del cine.

Encendían la TV, empezaba la carta de ajuste y todos se ponían contentos diciendo ``Ya empieza! ya empieza!'', aunque solo estuvieran esas barras estáticas. Echaban dibujos animados para los niños a la tarde, pero luego a la noche salía el telediario o películas para los mayores.

Cuando una película era para mayores, ponían dos rombos o incluso tres cuando era muy picante. A veces ponían un rombo, que quería decir que podían verlo los niños, que no era una cosa muy desagradable.



\section*{VIII.\\1972, enfermedad.}

El 28 de septiembre de 1972 yo me puse enferma. Estuve en cama un mes y no me daban con lo que tenía. El 28 de octubre me tuvieron que ingresar en el actual Hospital Xeral, que antes se llamaba Almirante Vierna. Le enviaron un telegrama a Antonio y vino a España para acompañarme. Cuando me hicieron los análisis dio tifus y era contagioso. Me llevaron para el hospital Municipal de las Camelias\footnote{Hoy Nicolás Peña.}, en Vigo. Allí estuve hasta el 16 de noviembre que me dieron el alta y volví para casa.


\section*{IX.\\1973, nacimiento de las hijas.}

En ese viaje de Antonio a España me quedé embarazada otra vez, prueba de que ya estaba completamente recuperada. Tuve una niña, que nació el 8 de agosto de 1973. Esta niña se llama Marina Teresa y sus padrinos fueron Jaime y Toñita. Cuando Antonio llegó a junto mía al hospital le pregunté ``Ya registraste a la niña?'' y le pregunté que qué nombre le había puesto, a lo que contestó ``Marina Teresa'' y yo le dije ``Malo raio che parta, non había outro nome máis feo?''.

En el 75 Antonio se vino definitivamente para España. Ya no se volvió a marchar de aquí. Buscó trabajo y estábamos todos juntos.

A los tres años y medio vino otra niña, María Montserrat, que nació el 16 de febrero de 1977. Los padrinos son su hermano José Antonio y Marisa Acevedo Márquez. No querían ponerle María porque mi madrastra era mala, pero al final se lo puso, y ahora le llamamos Montse. Le puso dos nombres a todos, porque era lo que se hacía en esa época.

\section*{X.\\1985, muerte de mi madrastra.}

El 29 de diciembre de 1985 murió mi tía, que tenía dos hermanos, Ramón y Horacio. Ellos estaban casados, Ramón tenía tres hijos, que se llamaban José Ramón, Emilio y Flora; y Horacio ninguno. La mujer de Ramón era portuguesa y se llamaba María, la de Horacio era vasca y se llamaba Leonor, a la que todos llamábamos Leo.

\section*{XI.\\1993, viaje a Argentina.}

Yo siempre tuve ganas de ir a Argentina, pero no podía. Decía: ``Voy a morir y no voy a ir a la Argentina a conocer a mis tíos''.
 
En Reyes de 1993 mi hijo Pepe me dijo ``Papá Noel te regala el viaje a Argentina'', entonces sus hermanos dijeron: ``Si tu le pagas el viaje a Mamá nosotros se lo pagamos a Papá y así van los dos''. 

Antonio no quería irse tanto tiempo porque teníamos mucho ganado y no se quedaba tranquilo dejándolo en manos de los chicos. Al escuchar esto, Evaristo se ofreció a cuidar a los animales y el campo, entonces Antonio pensó que sería buena idea. Un día de labor Evaristo llegó borracho a trabajar, así que perdió toda la confianza por parte de Antonio y vio que era necesario quedarse y encargarse él mismo. Entonces fui yo sola.

Me marché el 15 de enero de 1993. Todo fue muy bien, y a pesar de echar de menos a mi familia, me lo pasé bien.
 
Esta oferta del viaje estaba planteada para estar tres meses en Argentina, pero como a mí me parecía demasiado tiempo pensé que sería mejor estar solamente dos meses.

El primer mes que estuvimos en Buenos Aires me sentía muy sola, ya que mi familia estaba todo el día fuera de casa trabajando. Lloré mucho ese primer mes. No quería salir de casa porque no conocía a nadie y sentía que me podía perder, entre otros peligros.

El segundo mes se tomaron vacaciones y nos fuimos todos a Mar del Plata. Allí estábamos todos los días en la playa y estábamos en familia. Me sentí mucho más feliz y amparada. A las 18h echaban una novela en la radio entonces nos recogíamos para ir a escucharla en casa.

El 17 de marzo, dos meses después de haber llegado a Buenos Aires, volví a Galicia.



\section*{XII.\\1995-98, hijos que se van a Vielha.}

Yo tenía campos que heredé de mi abuelo materno. Vendimos uno que se llamaba A Lagoa y con el dinero compramos la casita que antes era de mis abuelos. En el reparto de la herencia, esta casita le había tocado a mi tío Pepe, que posteriormente él se la vendería a Font cuando se fue para Argentina.

Construímos un piso más encima de ella. Desde el 19 de abril del 1995, cuando Javi se va a vivir a Madrid, Antonio y yo estuvimos viviendo en la nueva planta superior.

Hoy en día, la casita de abajo es en la que yo vivo, y Montse vive con su familia en la parte de arriba.

El 4 de julio del 1995 se fue Marina para Vielha, el primero de agosto se fue Noli. Cuando llegaron las Navidades vinieron todos. Después de las fiestas, se volvieron Javi y Marina por donde habían venido, pero Noli no se marchó por que ya tenía novia y no le dejaba hacerlo. 

En mayo del 1996 también se fue Xe con idea de volver más tarde para ir a trabajar al monte donde estaba trabajando en ese momento. Pero como se encontraba bien en Cataluña, tenía trabajo y se ganaba bien la vida, decidió quedarse. A día de hoy es el único de mis hijos que sigue viviendo fuera de Galicia.

En el 1998 se fue Javi para Vielha y después se fue Mari Cruz, su novia. El 15 de mayo del 1999 se casaron, y el 13 de junio del 1999 nació Marta, su primera hija.


\section*{XIII.\\2005, intercambio de casas.}

En el 2005 nos vinimos para el piso de abajo porque el abuelo se puso enfermo y nos quedamos solos. Montse vivía abajo con su familia y ya eran cuatro, entonces cambiamos las casas. Ellas tendrían más espacio arriba y para el abuelo sería más fácil no tener que subir las escaleras.


\section*{XIV.\\Hijos y nietos}

El 27 de noviembre (de 1992?) se casó Pepe con su novia Goretti. El 11 de enero de 1996 nació Miguel su primer hijo, y el 6 de abril de 1998 nació David.

El 13 de junio de 1999, día de San Antonio, nació Marta, hija de Javi y Cruz. El 1 de mayo del 2000 volvió Marina de Vielha. El 20 de mayo del mismo año se casó Noli y el 6 de agosto del 2003 nació Lucía; la primera hija de Noli y Sonia. Y el 16 de agosto del mismo año nació Diego, el segundo hijo de Javi y Mari Cruz.

Pedro se quedó en Viella y allí conoció a Luisa, se enamoraron y se fueron a vivir juntos. Estuvieron así varios años y en el 2005 tuvieron un niño, Daniel se llamó. En el 2009 tuvieron a la niña que se llamó Gala.

En enero del 2002 Marina empezó a arreglar la casa de la carretera donde nací yo y donde se criaron mis hijos. Esa casa se levantó por primera vez en el año 1934, y ahora es reconstruida en el 2002.

Montse también arregló el bajo de la casa que le compramos a Font y se fue a vivir con Juanjo el 28 de febrero del 2002. El 13 de noviembre del 2003 nació María, su primera hija.

El 28 de junio vinieron Javi y la familia, y estaban viviendo de alquiler mientras se construía la casa en La Torre, Paramos.

El 28 de diciembre bautizaron a María, y los padrinos fueron mi hija Marina y el hermano de Juanjo, Héctor.

El 10 de abril del 2008 nació Ana, la segunda hija de Montse y Juanjo, y la bautizaron el 25 de Mayo del 2008. Los padrinos fueron Vanesa Barbosa y Manuel, otro hermano de Juanjo que ya falleció. El 13 de febrero de 2006 nació Lucas y el 5 de marzo de 2007 nació Helena, hijos de Pepe y Goretti. El 20 de noviembre del 2010 nació Manuel, y el 11 de diciembre de 2007 Claudia, hijos de Noli y Sonia. El 31 de diciembre de 2009 nació Gala, la hija de Pedro y Luisa. Así que tenemos 13 nietos y estamos muy contentos y muy orgullosos de ellos.


\section*{XV.\\Otras aventuras de la vida.}


El 29 de Mayo del 2005 Miguel el hijo de Pepe y Goretti hizo la Primera Comunión en la Catedral de Tui. El día 27 había fallecido mi tío Miguel, que había sido ingresado el día 20 de mayo. Lo enterraban el día 29, y después del banquete tuvimos que marchar a casa para cambiarnos e ir al entierro, nos trajo Montse. Al llegar a casa los perros se engancharon a pelear, Montse los fue a separar y la perra le mordió en una pierna, tuvo que llamar a su hermana para que la llevara a urgencias y a nosotros al tanatorio, y luego al cementerio.

En el mes de junio del 2007 hizo la Primera Comunión David, el segundo hijo de Pepe y Goretti. Otro día que pasamos la familia juntos. Eso es muy bonito, estar todos juntos con los hijos y los nietos.

Antonio y yo íbamos todos los veranos a Vielha a pasar unos días con los hijos que estaban allí. Antonio se sentía mal de una cadera y fue al médico, le dijo que tenían que operarle y lo pusieron en lista de espera, pero en un viaje a Vielha se lo contó a nuestro hijo y él habló con un médico cirujano que era amigo suyo, el Doctor Rivas. Le preguntó si podía operar a su padre, y él aceptó. Dijo que fuéramos hasta allá, que se empadronara le operaría. Fuimos para Vielha en el mes de julio del 2010, en el que lo iban a operar, pero resulta que estaban de vacaciones. Le pusieron fecha para el 19 de agosto, pero Antonio cogió un catarro muy fuerte y no lo podían operar así y tenían que curárselo, con lo cual le dieron fecha para el 30 de septiembre. Después de la operación, estuvimos en Vielha hasta que se recuperó y volvimos el 10 de Noviembre del 2010.

En octubre del 2012 le dieron cuatro infartos cerebrales y estuvo ingresado en el Hospital del Meixoeiro. Llegó a estar muy mal, pero se pudo recuperar un poco y llegar a encontrarse bastante mejor, incluso a defenderse bien. Yo lo ayudaba a vestirse, a acostarse, a ducharse. Él se lavaba, se peinaba y se afeitaba sólo; también comía por su mano. Si no llovía salíamos a dar un paseo. Se entretenía con los nietos porque no podía trabajar, y así pasaba el tiempo. Nuestros nietos nos entretienen tanto a él como a mí.

El día 8 de marzo de 2014 celebramos las bodas de oro, con misa y banquete. Comimos en el restaurante El Volante y pasamos un día muy feliz, con los hijos, las nueras, el yerno, y los nietos. Un día para no olvidar. Pensé para mí: ``Otro así no lo vamos a pasar, porque otros 50 años no vamos a vivir''. 

En las bodas de oro tuvimos sorpresas muy bonitas: por la mañana cuando fui a la peluquería me regalaron un ramo de flores; después, al llegar a casa, mi hija me trajo el ramo de novia y el de la solapa para el padre; al llegar a la iglesia estaban todos esperando y al entrar cantaba la coral del Centro Socio-comunitario de Tui. Después de la misa tenía otra sorpresa, pues las chicas del grupo Espenicelo me regalaron otro ramo de flores. Y al acabar la misa fuimos a comer al restaurante y a pasarlo bien el resto del día todos juntos.

El 22 de mayo de 2014 estuvo aquí Laura la portuguesa, porque vino a Portugal a limpiar la casa para cuando viniera Ana María de Argentina, y se acercó a Tui para hacernos una visita. El lunes 26 se marchó para Francia.

El 16 de junio llegaron a Portugal Ana María y Jani, y el 17 vinieron a hacernos una visita y comieron aquí. Después se fueron para Portugal porque dormían en casa de su prima Laura, ya que yo aquí no tengo sitio. Todos los días venían y salíamos con ellos.

El 25 de julio llegaron a Portugal Laura, mi prima y su nieta Lucía. El mismo viernes vinieron a mi casa a hacernos una visita. El domingo 27 vinieron a comer a mi casa Laura, su nieta, Laura, Francisco, y Delfina su hija. Mi prima y su nieta se quedaron aquí hasta el 17 de agosto, porque querían ver la procesión de San Roque, participar en ella y que Lucía viera como son aquí las fiestas porque en Argentina no son como en Galicia. Quedé muy contenta de que vinieran.

Ese año 2014 hubo de todo: cosas buenas y cosas malas, pero hay que ser fuerte y aguantar lo que venga, aunque a veces una no tenga ganas.

Bueno, me parece que con esto resumí un poco mi vida. Me despido de mis hijos, nietos y marido. A ver si los años que me quedan los termino mejor.



\begin{center}
\textbf{Un beso para todos}\\
\textbf{con mucho cariño.}
\end{center}




\end{document}
